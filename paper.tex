\documentclass[12pt,]{article}
\usepackage{lmodern}
\usepackage{setspace}
\setstretch{1.2}
\usepackage{amssymb,amsmath}
\usepackage{ifxetex,ifluatex}
\usepackage{fixltx2e} % provides \textsubscript
\ifnum 0\ifxetex 1\fi\ifluatex 1\fi=0 % if pdftex
  \usepackage[T1]{fontenc}
  \usepackage[utf8]{inputenc}
\else % if luatex or xelatex
  \ifxetex
    \usepackage{mathspec}
  \else
    \usepackage{fontspec}
  \fi
  \defaultfontfeatures{Ligatures=TeX,Scale=MatchLowercase}
    \setmainfont[]{Times New Roman}
    \setsansfont[]{Times New Roman}
\fi
% use upquote if available, for straight quotes in verbatim environments
\IfFileExists{upquote.sty}{\usepackage{upquote}}{}
% use microtype if available
\IfFileExists{microtype.sty}{%
\usepackage{microtype}
\UseMicrotypeSet[protrusion]{basicmath} % disable protrusion for tt fonts
}{}
\usepackage[margin=1in]{geometry}
\usepackage{hyperref}
\PassOptionsToPackage{usenames,dvipsnames}{color} % color is loaded by hyperref
\hypersetup{unicode=true,
            colorlinks=true,
            linkcolor=Maroon,
            citecolor=Blue,
            urlcolor=Blue,
            breaklinks=true}
\urlstyle{same}  % don't use monospace font for urls
\usepackage{color}
\usepackage{fancyvrb}
\newcommand{\VerbBar}{|}
\newcommand{\VERB}{\Verb[commandchars=\\\{\}]}
\DefineVerbatimEnvironment{Highlighting}{Verbatim}{commandchars=\\\{\}}
% Add ',fontsize=\small' for more characters per line
\usepackage{framed}
\definecolor{shadecolor}{RGB}{248,248,248}
\newenvironment{Shaded}{\begin{snugshade}}{\end{snugshade}}
\newcommand{\KeywordTok}[1]{\textcolor[rgb]{0.13,0.29,0.53}{\textbf{#1}}}
\newcommand{\DataTypeTok}[1]{\textcolor[rgb]{0.13,0.29,0.53}{#1}}
\newcommand{\DecValTok}[1]{\textcolor[rgb]{0.00,0.00,0.81}{#1}}
\newcommand{\BaseNTok}[1]{\textcolor[rgb]{0.00,0.00,0.81}{#1}}
\newcommand{\FloatTok}[1]{\textcolor[rgb]{0.00,0.00,0.81}{#1}}
\newcommand{\ConstantTok}[1]{\textcolor[rgb]{0.00,0.00,0.00}{#1}}
\newcommand{\CharTok}[1]{\textcolor[rgb]{0.31,0.60,0.02}{#1}}
\newcommand{\SpecialCharTok}[1]{\textcolor[rgb]{0.00,0.00,0.00}{#1}}
\newcommand{\StringTok}[1]{\textcolor[rgb]{0.31,0.60,0.02}{#1}}
\newcommand{\VerbatimStringTok}[1]{\textcolor[rgb]{0.31,0.60,0.02}{#1}}
\newcommand{\SpecialStringTok}[1]{\textcolor[rgb]{0.31,0.60,0.02}{#1}}
\newcommand{\ImportTok}[1]{#1}
\newcommand{\CommentTok}[1]{\textcolor[rgb]{0.56,0.35,0.01}{\textit{#1}}}
\newcommand{\DocumentationTok}[1]{\textcolor[rgb]{0.56,0.35,0.01}{\textbf{\textit{#1}}}}
\newcommand{\AnnotationTok}[1]{\textcolor[rgb]{0.56,0.35,0.01}{\textbf{\textit{#1}}}}
\newcommand{\CommentVarTok}[1]{\textcolor[rgb]{0.56,0.35,0.01}{\textbf{\textit{#1}}}}
\newcommand{\OtherTok}[1]{\textcolor[rgb]{0.56,0.35,0.01}{#1}}
\newcommand{\FunctionTok}[1]{\textcolor[rgb]{0.00,0.00,0.00}{#1}}
\newcommand{\VariableTok}[1]{\textcolor[rgb]{0.00,0.00,0.00}{#1}}
\newcommand{\ControlFlowTok}[1]{\textcolor[rgb]{0.13,0.29,0.53}{\textbf{#1}}}
\newcommand{\OperatorTok}[1]{\textcolor[rgb]{0.81,0.36,0.00}{\textbf{#1}}}
\newcommand{\BuiltInTok}[1]{#1}
\newcommand{\ExtensionTok}[1]{#1}
\newcommand{\PreprocessorTok}[1]{\textcolor[rgb]{0.56,0.35,0.01}{\textit{#1}}}
\newcommand{\AttributeTok}[1]{\textcolor[rgb]{0.77,0.63,0.00}{#1}}
\newcommand{\RegionMarkerTok}[1]{#1}
\newcommand{\InformationTok}[1]{\textcolor[rgb]{0.56,0.35,0.01}{\textbf{\textit{#1}}}}
\newcommand{\WarningTok}[1]{\textcolor[rgb]{0.56,0.35,0.01}{\textbf{\textit{#1}}}}
\newcommand{\AlertTok}[1]{\textcolor[rgb]{0.94,0.16,0.16}{#1}}
\newcommand{\ErrorTok}[1]{\textcolor[rgb]{0.64,0.00,0.00}{\textbf{#1}}}
\newcommand{\NormalTok}[1]{#1}
\usepackage{longtable,booktabs}
\usepackage{graphicx,grffile}
\makeatletter
\def\maxwidth{\ifdim\Gin@nat@width>\linewidth\linewidth\else\Gin@nat@width\fi}
\def\maxheight{\ifdim\Gin@nat@height>\textheight\textheight\else\Gin@nat@height\fi}
\makeatother
% Scale images if necessary, so that they will not overflow the page
% margins by default, and it is still possible to overwrite the defaults
% using explicit options in \includegraphics[width, height, ...]{}
\setkeys{Gin}{width=\maxwidth,height=\maxheight,keepaspectratio}
\IfFileExists{parskip.sty}{%
\usepackage{parskip}
}{% else
\setlength{\parindent}{0pt}
\setlength{\parskip}{6pt plus 2pt minus 1pt}
}
\setlength{\emergencystretch}{3em}  % prevent overfull lines
\providecommand{\tightlist}{%
  \setlength{\itemsep}{0pt}\setlength{\parskip}{0pt}}
\setcounter{secnumdepth}{5}
% Redefines (sub)paragraphs to behave more like sections
\ifx\paragraph\undefined\else
\let\oldparagraph\paragraph
\renewcommand{\paragraph}[1]{\oldparagraph{#1}\mbox{}}
\fi
\ifx\subparagraph\undefined\else
\let\oldsubparagraph\subparagraph
\renewcommand{\subparagraph}[1]{\oldsubparagraph{#1}\mbox{}}
\fi

%%% Use protect on footnotes to avoid problems with footnotes in titles
\let\rmarkdownfootnote\footnote%
\def\footnote{\protect\rmarkdownfootnote}

%%% Change title format to be more compact
\usepackage{titling}

% Create subtitle command for use in maketitle
\newcommand{\subtitle}[1]{
  \posttitle{
    \begin{center}\large#1\end{center}
    }
}

\setlength{\droptitle}{-2em}

  \title{\vspace{1cm}Writing a reproducible paper\\
in R Markdown*\footnote{*Corresponding address:
  \href{mailto:mail@paulcbauer.eu}{\nolinkurl{mail@paulcbauer.eu}}.
  Acknowledgments: I am grateful to all those generous people that
  invest their time into open-source software.}\vspace{0.5cm}\\}
    \pretitle{\vspace{\droptitle}\centering\huge}
  \posttitle{\par}
    \author{Paul C. Bauer\\
Mannheim Centre for European Social Research}
    \preauthor{\centering\large\emph}
  \postauthor{\par}
      \predate{\centering\large\emph}
  \postdate{\par}
    \date{2 November, 2018\\[2\baselineskip]Feedback?\\
\href{mailto:mail@paulcbauer.eu}{\nolinkurl{mail@paulcbauer.eu}}\\}

\usepackage[english]{babel}
\usepackage[T1]{fontenc}
\usepackage{float}
\usepackage{lmodern}
\usepackage{graphicx}
\usepackage{url}
\usepackage{array}
\usepackage{setspace}
\usepackage{graphics}
%\usepackage{amssymb}
\usepackage{amsmath}
\usepackage{longtable}
\usepackage{natbib}
\usepackage{pdflscape}
\usepackage{caption}
\usepackage{subcaption}
\usepackage{fullpage}
\usepackage{multicol}
\usepackage{dcolumn}
\usepackage{lscape}
\usepackage{array}
\usepackage{color}
\usepackage{setspace}

\usepackage{cleveref}[2012/02/15]
\usepackage{fullpage}
\usepackage[charter]{mathdesign}
\usepackage{bm}
\usepackage{tabu}
\usepackage{wrapfig}


\crefformat{footnote}{#2\footnotemark[#1]#3}
\usepackage{color}
\usepackage{hyperref}

\newenvironment{tightcenter}{%
  \setlength\topsep{0pt}
  \setlength\parskip{0pt}
  \begin{center}
}{%
  \end{center}
}

%\setstretch{1.1}
% Dutch style of paragraph formatting
\parskip 5pt
\makeatletter
% Separation between lines
\doublerulesep 1pt

\usepackage[table]{xcolor}

\definecolor{lightred}{rgb}{0.7,0,0}
\definecolor{darkgreen}{rgb}{0,0.8,0}
\definecolor{lightblue}{rgb}{0,0,0.7}

\hypersetup{colorlinks,
  linkcolor = lightred,
  filecolor = lightred,
  urlcolor = lightblue,
  citecolor = lightred}

% Spaces in bibliography
\let\oldthebibliography=\thebibliography
\let\endoldthebibliography=\endthebibliography
\renewenvironment{thebibliography}[1]{%
  \begin{oldthebibliography}{#1}%
    \setlength{\parskip}{0ex}%
    \setlength{\itemsep}{0.1ex}%
  }%
  {%
  \end{oldthebibliography}%
}

\usepackage{array}
\newcolumntype{L}[1]{>{\raggedright\let\newline\\\arraybackslash\hspace{0pt}}m{#1}}
\newcolumntype{C}[1]{>{\centering\let\newline\\\arraybackslash\hspace{0pt}}m{#1}}
\newcolumntype{R}[1]{>{\raggedleft\let\newline\\\arraybackslash\hspace{0pt}}m{#1}}

\setlength{\parindent}{0pt} % no indentation
\usepackage{booktabs}
\usepackage{longtable}
\usepackage{array}
\usepackage{multirow}
\usepackage[table]{xcolor}
\usepackage{wrapfig}
\usepackage{float}
\usepackage{colortbl}
\usepackage{pdflscape}
\usepackage{tabu}
\usepackage{threeparttable}
\usepackage{threeparttablex}
\usepackage[normalem]{ulem}
\usepackage{makecell}

\usepackage{dcolumn}
\usepackage{color}

\usepackage{amsthm}
\newtheorem{theorem}{Theorem}[section]
\newtheorem{lemma}{Lemma}[section]
\theoremstyle{definition}
\newtheorem{definition}{Definition}[section]
\newtheorem{corollary}{Corollary}[section]
\newtheorem{proposition}{Proposition}[section]
\theoremstyle{definition}
\newtheorem{example}{Example}[section]
\theoremstyle{definition}
\newtheorem{exercise}{Exercise}[section]
\theoremstyle{remark}
\newtheorem*{remark}{Remark}
\newtheorem*{solution}{Solution}
\begin{document}
\maketitle
\begin{abstract}
\noindent\setstretch{1}The present paper provides a template for a
reproducable scientific paper written in R Markdown. Below I outline
some of the ``tricks''/code (e.g., referencing tables, sections etc.) I
had to figure out to produce this document. The underlying files which
produce this document can be downloaded
\href{https://drive.google.com/drive/folders/1zJP3cNPrHN-gj0rcmbHQgg-XA0hqDXdd?usp=sharing}{here}.
I think I got pretty far but there is always room for improvement and
more automatization, in parallel to the incredible developments in R and
Rstudio (bookdown etc.). My last paper is almost entirely based on R
Markdown (see Bauer \protect\hyperlink{ref-Bauer2018-hd}{2018} and
\href{https://dataverse.harvard.edu/dataset.xhtml?persistentId=doi:10.7910/DVN/DUMGLT}{reproduction
files}). I intend to update this file when I discover more convenient
code.\vspace{.8cm}
\end{abstract}

\clearpage

\renewcommand{\baselinestretch}{0.5}

\normalsize
\tableofcontents
\renewcommand{\baselinestretch}{1.1}\normalsize

\clearpage

\section{Why reproducible research (in
R)?}\label{why-reproducible-research-in-r}

Some arguments\ldots{}

\begin{itemize}
\tightlist
\item
  \textbf{Access}: Research is normally funded by taxpayers (researchers
  are also taxpayers). Hence, it should be freely accessible to everyone
  without any barriers, e.g., without requiring commercial software.
  Importantly, researchers from developing countries are even more
  dependent on free access to knowledge (Kirsop and Chan
  \protect\hyperlink{ref-Kirsop2005-ro}{2005}).
\item
  \textbf{Reproducability}: Even if you have written a study and
  analyzed the data yourself you will forget what you did after a few
  months. A fully reproducable setup will help you to trace back your
  own steps. Obviously, the same is true for other researchers who may
  want to understand your work and built on it. It may sound like a joke
  but why not aim for a document that can be used to reproduce your
  findings in in 500 years.
\item
  \textbf{Errors}: Manual steps in data analysis (e.g., manually
  copy/pasting values into a table etc.) may introduce errors. R
  Markdown allows you to \textbf{automatize} such steps and/or avoid
  them.
\item
  \textbf{Revisions}: Revising a paper takes much less time if you have
  all the code you need in one place, i.e., one \texttt{.rmd} file. For
  instance, if you decide to exclude a subset of your data you simply
  need to insert one line of your code at the beginning and everything
  is rebuilt/re-estimated automatically.
\end{itemize}

\section{Prerequesites}\label{prerequesites}

I assume that you are using R on a day-to-day basis. You may have even
started to work a little in R Markdown but you don't write your complete
paper in R Markdown. If you don't know what R Markdown is watch
\href{https://vimeo.com/178485416}{this short video}. Then\ldots{}

\begin{itemize}
\tightlist
\item
  \ldots{}install \href{https://www.r-project.org/}{R} and
  \href{https://www.rstudio.com/}{Rstudio} (most recent versions) (R
  Core Team \protect\hyperlink{ref-R2017}{2017}; RStudio Team
  \protect\hyperlink{ref-Rstudio2015}{2015}).
\item
  \ldots{}install \href{https://yihui.name/tinytex/}{tinytex}, a
  lightweight version of Tex Live (Allaire et al.
  \protect\hyperlink{ref-markdown2017}{2017}; Xie
  \protect\hyperlink{ref-tinytex}{2018}\protect\hyperlink{ref-tinytex}{b}).
\end{itemize}

\begin{Shaded}
\begin{Highlighting}[]
\KeywordTok{install.packages}\NormalTok{(}\KeywordTok{c}\NormalTok{(}\StringTok{'tinytex'}\NormalTok{, }\StringTok{'rmarkdown'}\NormalTok{))}
\NormalTok{tinytex}\OperatorTok{::}\KeywordTok{install_tinytex}\NormalTok{()}
\end{Highlighting}
\end{Shaded}

\begin{itemize}
\tightlist
\item
  \ldots{}install the packages below using the code below (Sievert et
  al. \protect\hyperlink{ref-plotly}{2017}; Xie
  \protect\hyperlink{ref-knitr3}{2014},
  \protect\hyperlink{ref-knitr2}{2015},
  \protect\hyperlink{ref-bookdown2}{2016},
  \protect\hyperlink{ref-bookdown1}{2017},
  \protect\hyperlink{ref-knitr1}{2018}\protect\hyperlink{ref-knitr1}{a};
  Zhu \protect\hyperlink{ref-kableextra}{2017}).
\end{itemize}

\begin{Shaded}
\begin{Highlighting}[]
\KeywordTok{install.packages}\NormalTok{(}\KeywordTok{c}\NormalTok{(}\StringTok{"rmarkdown"}\NormalTok{, }\StringTok{"knitr"}\NormalTok{, }\StringTok{"kableExtra"}\NormalTok{,}
                   \StringTok{"stargazer"}\NormalTok{, }\StringTok{"plotly"}\NormalTok{, }\StringTok{"knitr"}\NormalTok{,}
                   \StringTok{"bookdown"}\NormalTok{))}
\end{Highlighting}
\end{Shaded}

\begin{itemize}
\item
  \ldots{}download the 5 input files I created --- \texttt{paper.rmd},
  \texttt{header.tex}, \texttt{references.bib}, \texttt{data.csv} and
  \texttt{american-sociological-association.csl} --- from
  \href{https://drive.google.com/drive/folders/1zJP3cNPrHN-gj0rcmbHQgg-XA0hqDXdd?usp=sharing}{this
  folder}. Ignore the other files.
\item
  \ldots{}learn R and read about the other underlying components namely
  \href{https://en.wikipedia.org/wiki/Markdown}{Markdown},
  \href{https://rmarkdown.rstudio.com/lesson-1.html}{R Markdown} and
  \href{https://en.wikipedia.org/wiki/LaTeX}{Latex}.
\end{itemize}

\section{Basics: Input and output
files}\label{basics-input-and-output-files}

All the files you need to produce the present PDF file are the input
files\ldots{}

\begin{itemize}
\tightlist
\item
  \ldots{}a \texttt{paper.rmd} file (the underlying R Markdown file).
\item
  \ldots{}a \texttt{header.tex} file (a tex file for document
  structure).
\item
  \ldots{}a \texttt{references.bib} file (the bibliography).

  \begin{itemize}
  \tightlist
  \item
    I use paperpile to manage my references and export the \texttt{.bib}
    file into the folder that contains my \texttt{.rmd} file.
  \end{itemize}
\item
  \ldots{}a \texttt{data.csv} file (some raw data).
\item
  \ldots{} a \texttt{american-sociological-association.csl} file that
  defines the style of your bibliography.\footnote{You can download
    various citation style files from this webpage:
    \url{https://github.com/citation-style-language/styles}.}
\end{itemize}

\href{https://drive.google.com/drive/folders/1zJP3cNPrHN-gj0rcmbHQgg-XA0hqDXdd?usp=sharing}{Download
these files} and save them into a folder. Close R/Rstudio and directly
open \texttt{paper.rmd} with RStudio. Doing so assures that the working
directory is set to the folder that contains \texttt{paper.rmd} and the
other files.\footnote{You can always check your working directory in R
  with \texttt{getwd()}.}

Once you run/compile the \texttt{paper.rmd} file in Rstudio it creates
mainly two output files:

\begin{itemize}
\tightlist
\item
  \texttt{paper.tex}
\item
  \texttt{paper.pdf} (the one you are reading right now)
\end{itemize}

In addition, there may be files that you generate and store locally in
the folder during the compilation process. This is the case for some of
the Plotly graphs below.

Ideally, we can simply provide others with a \texttt{zip} folder that
contains both our input files and our output files. Then it's possible
to reproduce the process from managing/analyzing some raw data to
producing the final scientific article.

Below we always display the R code in the chunks that produce the
output. In your paper you will normally only present outputs (e.g.,
tables, figures etc.) by choosing the chunk option ``Show output only''
in R Studio. The chunk commands itself are not displayed but they do
matter for referencing etc. So simply orient yourself at the underlying
\texttt{paper.rmd} file.

\section{Referencing within your
document}\label{referencing-within-your-document}

To see how referencing works simply see the different examples for
figures, tables and sections below. For instance in Section
\ref{sec:tables} you can find different ways of referencing tables. The
code of the underlying \texttt{paper.rmd} will show you how I referenced
Section \ref{sec:tables} right here namely with
`\texttt{Section\ \textbackslash{}@ref(sec:tables)}'.

\section{Software versioning}\label{software-versioning}

Software changes and gets updated, especially with an active developer
community like that of R. Luckily you can always access
\href{https://cran.r-project.org/bin/windows/base/old/}{old versions of
R} and old version of R packages in
\href{https://cran.r-project.org/src/contrib/Archive/}{the archive}. In
the archive you need to choose a particular package, e.g dplyr and
search for the right version, e.g., \texttt{dplyr\_0.2.tar.gz}. Then
insert the path in the following function:
\texttt{install.packages("https://....../dplyr\_0.2.tar.gz",\ repos=NULL,\ type="source")}.
Ideally, however, results will be simply reproducible in the most
current R and package versions.

I would recommend to use the command below and copy/paste the output
into the first R chunk in your \texttt{paper.rmd} file. Normally we
would do this when we compile/generate the paper for the last time. For
more advanced tools see
\href{https://rstudio.github.io/packrat/}{packrat}.

\begin{Shaded}
\begin{Highlighting}[]
\KeywordTok{cat}\NormalTok{(}\KeywordTok{paste}\NormalTok{(}\StringTok{"#"}\NormalTok{, }\KeywordTok{capture.output}\NormalTok{(}\KeywordTok{sessionInfo}\NormalTok{()), }\StringTok{"}\CharTok{\textbackslash{}n}\StringTok{"}\NormalTok{, }\DataTypeTok{collapse =}\StringTok{""}\NormalTok{)) }
\end{Highlighting}
\end{Shaded}

\begin{verbatim}
## # R version 3.5.1 (2018-07-02) 
## # Platform: x86_64-w64-mingw32/x64 (64-bit) 
## # Running under: Windows 10 x64 (build 17134) 
## #  
## # Matrix products: default 
## #  
## # locale: 
## # [1] LC_COLLATE=English_Germany.1252  LC_CTYPE=English_Germany.1252    
## # [3] LC_MONETARY=English_Germany.1252 LC_NUMERIC=C                     
## # [5] LC_TIME=English_Germany.1252     
## #  
## # attached base packages: 
## # [1] stats     graphics  grDevices utils     datasets  methods   base      
## #  
## # loaded via a namespace (and not attached): 
## #  [1] compiler_3.5.1  backports_1.1.2 magrittr_1.5    bookdown_0.7    
## #  [5] rprojroot_1.3-2 tools_3.5.1     htmltools_0.3.6 yaml_2.2.0      
## #  [9] Rcpp_0.12.19    stringi_1.1.7   rmarkdown_1.10  knitr_1.20      
## # [13] stringr_1.3.1   xfun_0.3        digest_0.6.18   evaluate_0.12
\end{verbatim}

\begin{Shaded}
\begin{Highlighting}[]
  \CommentTok{# or use message() instead of cat()}
\end{Highlighting}
\end{Shaded}

\section{Data}\label{data}

\subsection{Import}\label{import}

\begin{Shaded}
\begin{Highlighting}[]
\NormalTok{data <-}\StringTok{ }\KeywordTok{read.csv}\NormalTok{(}\StringTok{"data.csv"}\NormalTok{)}
\KeywordTok{head}\NormalTok{(data)}
\end{Highlighting}
\end{Shaded}

\begin{verbatim}
##   X speed dist
## 1 1     4    2
## 2 2     4   10
## 3 3     7    4
## 4 4     7   22
## 5 5     8   16
## 6 6     9   10
\end{verbatim}

\subsection{Putting your entire data into the .rmd
file}\label{putting-your-entire-data-into-the-.rmd-file}

Applying the function \texttt{dput()} to an object gives you the code
needed to reproduce that object. So you could paste that code into your
\texttt{.rmd} file if you don't want to have extra data files. This
makes sense were data files are small.

\begin{Shaded}
\begin{Highlighting}[]
\KeywordTok{dput}\NormalTok{(data)}
\end{Highlighting}
\end{Shaded}

\begin{verbatim}
## structure(list(X = 1:50, speed = c(4L, 4L, 7L, 7L, 8L, 9L, 10L, 
## 10L, 10L, 11L, 11L, 12L, 12L, 12L, 12L, 13L, 13L, 13L, 13L, 14L, 
## 14L, 14L, 14L, 15L, 15L, 15L, 16L, 16L, 17L, 17L, 17L, 18L, 18L, 
## 18L, 18L, 19L, 19L, 19L, 20L, 20L, 20L, 20L, 20L, 22L, 23L, 24L, 
## 24L, 24L, 24L, 25L), dist = c(2L, 10L, 4L, 22L, 16L, 10L, 18L, 
## 26L, 34L, 17L, 28L, 14L, 20L, 24L, 28L, 26L, 34L, 34L, 46L, 26L, 
## 36L, 60L, 80L, 20L, 26L, 54L, 32L, 40L, 32L, 40L, 50L, 42L, 56L, 
## 76L, 84L, 36L, 46L, 68L, 32L, 48L, 52L, 56L, 64L, 66L, 54L, 70L, 
## 92L, 93L, 120L, 85L)), class = "data.frame", row.names = c(NA, 
## -50L))
\end{verbatim}

You can then insert the dput output in your \texttt{.rmd} as below.

\begin{Shaded}
\begin{Highlighting}[]
\NormalTok{data <-}\StringTok{ }\KeywordTok{structure}\NormalTok{(}\KeywordTok{list}\NormalTok{(}\DataTypeTok{X =} \DecValTok{1}\OperatorTok{:}\DecValTok{50}\NormalTok{, }\DataTypeTok{speed =} \KeywordTok{c}\NormalTok{(4L, 4L, 7L, 7L, 8L, 9L, 10L, }
\NormalTok{10L, 10L, 11L, 11L, 12L, 12L, 12L, 12L, 13L, 13L, 13L, 13L, 14L, }
\NormalTok{14L, 14L, 14L, 15L, 15L, 15L, 16L, 16L, 17L, 17L, 17L, 18L, 18L, }
\NormalTok{18L, 18L, 19L, 19L, 19L, 20L, 20L, 20L, 20L, 20L, 22L, 23L, 24L, }
\NormalTok{24L, 24L, 24L, 25L), }\DataTypeTok{dist =} \KeywordTok{c}\NormalTok{(2L, 10L, 4L, 22L, 16L, 10L, 18L, }
\NormalTok{26L, 34L, 17L, 28L, 14L, 20L, 24L, 28L, 26L, 34L, 34L, 46L, 26L, }
\NormalTok{36L, 60L, 80L, 20L, 26L, 54L, 32L, 40L, 32L, 40L, 50L, 42L, 56L, }
\NormalTok{76L, 84L, 36L, 46L, 68L, 32L, 48L, 52L, 56L, 64L, 66L, 54L, 70L, }
\NormalTok{92L, 93L, 120L, 85L)), }
\DataTypeTok{class =} \StringTok{"data.frame"}\NormalTok{, }\DataTypeTok{row.names =} \KeywordTok{c}\NormalTok{(}\OtherTok{NA}\NormalTok{, }
\OperatorTok{-}\NormalTok{50L))}
\end{Highlighting}
\end{Shaded}

\section{Tables}\label{sec:tables}

Producing good tables and referencing these tables within a R Markdown
PDF has been a hassle but got much better. Examples that you may use are
shown below. The way you reference tables is slightly different, e.g.,
for \texttt{stargazer} the label is contained in the function, for
\texttt{kable} it's contained in the chunk name.

\subsection{stargazer(): Summary and regression
tables}\label{stargazer-summary-and-regression-tables}

Table \ref{tab1} shows summary stats of your data.\footnote{To reference
  the table where you set the identifier in the stargazer function you
  only need to use the actual label, i.e., ´tab1´.} I normally use
\texttt{stargazer()} (Hlavac
\protect\hyperlink{ref-hlavac2013stargazer}{2013}) which offers extreme
flexibility regarding table output (see \texttt{?stargazer}).

\begin{Shaded}
\begin{Highlighting}[]
\KeywordTok{library}\NormalTok{(stargazer)}
\KeywordTok{stargazer}\NormalTok{(cars, }
          \DataTypeTok{title =} \StringTok{"Summary table with stargazer"}\NormalTok{,}
          \DataTypeTok{label=}\StringTok{"tab1"}\NormalTok{, }
          \DataTypeTok{table.placement =} \StringTok{"H"}\NormalTok{, }
          \DataTypeTok{header=}\OtherTok{FALSE}\NormalTok{)}
\end{Highlighting}
\end{Shaded}

\begin{table}[H] \centering 
  \caption{Summary table with stargazer} 
  \label{tab1} 
\begin{tabular}{@{\extracolsep{5pt}}lccccccc} 
\\[-1.8ex]\hline 
\hline \\[-1.8ex] 
Statistic & \multicolumn{1}{c}{N} & \multicolumn{1}{c}{Mean} & \multicolumn{1}{c}{St. Dev.} & \multicolumn{1}{c}{Min} & \multicolumn{1}{c}{Pctl(25)} & \multicolumn{1}{c}{Pctl(75)} & \multicolumn{1}{c}{Max} \\ 
\hline \\[-1.8ex] 
speed & 50 & 15.400 & 5.288 & 4 & 12 & 19 & 25 \\ 
dist & 50 & 42.980 & 25.769 & 2 & 26 & 56 & 120 \\ 
\hline \\[-1.8ex] 
\end{tabular} 
\end{table}

Table \ref{tab2} shows the output for a regression table. Make sure you
name all your models and explicitly refer to model names (M1, M2 etc.)
in the text.

\begin{Shaded}
\begin{Highlighting}[]
\KeywordTok{library}\NormalTok{(stargazer)}
\NormalTok{model1 <-}\StringTok{ }\KeywordTok{lm}\NormalTok{(speed }\OperatorTok{~}\StringTok{ }\NormalTok{dist, }\DataTypeTok{data =}\NormalTok{ cars)}
\NormalTok{model2 <-}\StringTok{ }\KeywordTok{lm}\NormalTok{(speed }\OperatorTok{~}\StringTok{ }\NormalTok{dist, }\DataTypeTok{data =}\NormalTok{ cars)}
\NormalTok{model3 <-}\StringTok{ }\KeywordTok{lm}\NormalTok{(dist }\OperatorTok{~}\StringTok{ }\NormalTok{speed, }\DataTypeTok{data =}\NormalTok{ cars)}
\KeywordTok{stargazer}\NormalTok{(model1, model2, model3,}
          \DataTypeTok{title =} \StringTok{"Regression table with stargazer"}\NormalTok{,}
          \DataTypeTok{label=}\StringTok{"tab2"}\NormalTok{, }
          \DataTypeTok{table.placement =} \StringTok{"H"}\NormalTok{, }
          \DataTypeTok{column.labels =} \KeywordTok{c}\NormalTok{(}\StringTok{"M1"}\NormalTok{, }\StringTok{"M2"}\NormalTok{, }\StringTok{"M3"}\NormalTok{),}
          \DataTypeTok{model.numbers =} \OtherTok{FALSE}\NormalTok{,}
          \DataTypeTok{header=}\OtherTok{FALSE}\NormalTok{)}
\end{Highlighting}
\end{Shaded}

\begin{table}[H] \centering 
  \caption{Regression table with stargazer} 
  \label{tab2} 
\begin{tabular}{@{\extracolsep{5pt}}lccc} 
\\[-1.8ex]\hline 
\hline \\[-1.8ex] 
 & \multicolumn{3}{c}{\textit{Dependent variable:}} \\ 
\cline{2-4} 
\\[-1.8ex] & \multicolumn{2}{c}{speed} & dist \\ 
 & M1 & M2 & M3 \\ 
\hline \\[-1.8ex] 
 dist & 0.166$^{***}$ & 0.166$^{***}$ &  \\ 
  & (0.017) & (0.017) &  \\ 
  & & & \\ 
 speed &  &  & 3.932$^{***}$ \\ 
  &  &  & (0.416) \\ 
  & & & \\ 
 Constant & 8.284$^{***}$ & 8.284$^{***}$ & $-$17.579$^{**}$ \\ 
  & (0.874) & (0.874) & (6.758) \\ 
  & & & \\ 
\hline \\[-1.8ex] 
Observations & 50 & 50 & 50 \\ 
R$^{2}$ & 0.651 & 0.651 & 0.651 \\ 
Adjusted R$^{2}$ & 0.644 & 0.644 & 0.644 \\ 
Residual Std. Error (df = 48) & 3.156 & 3.156 & 15.380 \\ 
F Statistic (df = 1; 48) & 89.567$^{***}$ & 89.567$^{***}$ & 89.567$^{***}$ \\ 
\hline 
\hline \\[-1.8ex] 
\textit{Note:}  & \multicolumn{3}{r}{$^{*}$p$<$0.1; $^{**}$p$<$0.05; $^{***}$p$<$0.01} \\ 
\end{tabular} 
\end{table}

\subsection{kable() and kable\_styling()}\label{kable-and-kable_styling}

Another great function is \texttt{kable()} (\texttt{knitr} package) in
combination with \texttt{kableExtra}. Table \ref{tab:tab3} provides an
example.\footnote{To reference the table produced by the chunk you need
  to add ´tab:´ to the chunk name, i.e., ´tab:tab3´.} Again you can
modify so many things in both the \texttt{kable()} and the
\texttt{kable\_styling()} function. See
\href{https://haozhu233.github.io/kableExtra/awesome_table_in_pdf.pdf}{this
overview} of all the kable stylings that are possible provided by the
package author himself.

\begin{Shaded}
\begin{Highlighting}[]
\KeywordTok{library}\NormalTok{(knitr)}
\KeywordTok{library}\NormalTok{(kableExtra)}
\KeywordTok{kable}\NormalTok{(cars[}\DecValTok{1}\OperatorTok{:}\DecValTok{10}\NormalTok{,], }\DataTypeTok{row.names =} \OtherTok{TRUE}\NormalTok{, }
      \DataTypeTok{caption =} \StringTok{'Table with kable() and kablestyling()'}\NormalTok{, }
      \DataTypeTok{format =} \StringTok{"latex"}\NormalTok{, }\DataTypeTok{booktabs =}\NormalTok{ T) }\OperatorTok
\StringTok{        }\KeywordTok{kable_styling}\NormalTok{(}\DataTypeTok{full_width =}\NormalTok{ T, }
                      \DataTypeTok{latex_options =} \KeywordTok{c}\NormalTok{(}\StringTok{"striped"}\NormalTok{, }
                                        \StringTok{"scale_down"}\NormalTok{,}
                                        \StringTok{"HOLD_position"}\NormalTok{),}
                      \DataTypeTok{font_size =} \DecValTok{10}\NormalTok{)}
\end{Highlighting}
\end{Shaded}

\rowcolors{2}{gray!6}{white}

\begin{table}[H]

\caption{\label{tab:tab3}Table with kable() and kablestyling()}
\centering
\resizebox{\linewidth}{!}{
\fontsize{10}{12}\selectfont
\begin{tabu} to \linewidth {>{\raggedright}X>{\raggedleft}X>{\raggedleft}X}
\hiderowcolors
\toprule
  & speed & dist\\
\midrule
\showrowcolors
1 & 4 & 2\\
2 & 4 & 10\\
3 & 7 & 4\\
4 & 7 & 22\\
5 & 8 & 16\\
\addlinespace
6 & 9 & 10\\
7 & 10 & 18\\
8 & 10 & 26\\
9 & 10 & 34\\
10 & 11 & 17\\
\bottomrule
\end{tabu}}
\end{table}

\rowcolors{2}{white}{white}

\section{Inline code \& results}\label{inline-code-results}

Reproduction reaches new heights when you work with inline code. For
instance, you can automatize the display of certain coefficients within
the text. An example is to include estimates, e.g., the coefficient of
\texttt{dist} of the model we ran above.
\texttt{\textasciigrave{}r\ round(coef(model1){[}2{]},\ 2)\textasciigrave{}}
will insert the coefficient as follows: 0.17. Or
\texttt{\textasciigrave{}r\ 3\ +\ 7\textasciigrave{}} will insert a 10
in the text.\\
Inline code/results that depend on earlier objects in your document will
automatically be updated once you change those objects. For instance,
imagine a reviewer asks you to omit certain observations from your
sample. You can simply do so in the beginning of your code and push play
subsequently.. at time you might have to set \texttt{cache\ =\ FALSE} at
the beginning so that all the code chungs are rerun.\\
Researchers often avoid referring to results intext etc. because you
easily forget to change them when revising a manuscript. At the same it
can make an article much more informative and easier to read, e.g., if
you discuss a coefficient in the text you can directly show it in the
section in which you discuss it. Inline code allows you to do just that.
R Markdown allows you to that do so in a reproducible and automatized
manner.

\section{Figures}\label{figures}

\subsection{R base graphs}\label{r-base-graphs}

Inserting figures can be slightly more complicated. Ideally, we would
produce and insert them directly in the \texttt{.rmd} file. It's
relatively simple to insert R base graphs as you can see in Figure
\ref{fig:fig-1}.

\begin{Shaded}
\begin{Highlighting}[]
\KeywordTok{plot}\NormalTok{(cars}\OperatorTok{$}\NormalTok{speed, cars}\OperatorTok{$}\NormalTok{dist)}
\end{Highlighting}
\end{Shaded}

\begin{figure}[H]

{\centering \includegraphics{paper_files/figure-latex/fig-1-1} 

}

\caption{Scatterplot of Speed and Distance}\label{fig:fig-1}
\end{figure}

But it turns out that it doesn't always work so well.

\subsection{ggplot2 graphs}\label{ggplot2-graphs}

Same is true for ggplot2 as you can see in Figure \ref{fig:fig-2}.

\begin{Shaded}
\begin{Highlighting}[]
\NormalTok{mtcars}\OperatorTok{$}\NormalTok{cyl <-}\StringTok{ }\KeywordTok{as.factor}\NormalTok{(mtcars}\OperatorTok{$}\NormalTok{cyl) }\CommentTok{# Convert cyl to factor}
\KeywordTok{library}\NormalTok{(ggplot2)}
\KeywordTok{ggplot}\NormalTok{(mtcars, }\KeywordTok{aes}\NormalTok{(}\DataTypeTok{x=}\NormalTok{wt, }\DataTypeTok{y=}\NormalTok{mpg, }\DataTypeTok{shape=}\NormalTok{cyl)) }\OperatorTok{+}\StringTok{ }\KeywordTok{geom_point}\NormalTok{() }\OperatorTok{+}
\StringTok{  }\KeywordTok{labs}\NormalTok{(}\DataTypeTok{x=}\StringTok{"Weight (lb/1000)"}\NormalTok{, }\DataTypeTok{y =} \StringTok{"Miles/(US) gallon"}\NormalTok{, }
       \DataTypeTok{shape=}\StringTok{"Number of }\CharTok{\textbackslash{}n}\StringTok{ Cylinders"}\NormalTok{) }\OperatorTok{+}\StringTok{ }\KeywordTok{theme_classic}\NormalTok{()}
\end{Highlighting}
\end{Shaded}

\begin{figure}[H]

{\centering \includegraphics{paper_files/figure-latex/fig-2-1} 

}

\caption{Miles per gallon according to the weight}\label{fig:fig-2}
\end{figure}

\subsection{Plotly graphs}\label{plotly-graphs}

Plotly is a popular graph engine that let's you also produce interactive
graphs that you can embed in html webpages or documents (e.g., see
\href{https://paulcbauer.shinyapps.io/visualizing-causal-scenarios/}{here}).
I am a big fan, however, right now there is no easy, automatic way to
insert high resolution Plotly graphs into your R Markdown PDF. See
Figure \ref{fig:fig-3} below.\\
There seems to be a new engine for exporting graphs locally without
relying on the plotly server --- which you wanna avoid for a properly
reproducible PDF --- called orca (\texttt{?orca}). However, the
\href{https://github.com/plotly/orca\#installation}{installation} is too
challenging for the average user at the moment.

\begin{Shaded}
\begin{Highlighting}[]
\KeywordTok{library}\NormalTok{(plotly)}
\KeywordTok{plot_ly}\NormalTok{(cars, }\DataTypeTok{type =} \StringTok{"scatter"}\NormalTok{, }\DataTypeTok{mode=}\StringTok{"markers"}\NormalTok{,}
        \DataTypeTok{x=}\OperatorTok{~}\NormalTok{speed, }
        \DataTypeTok{y=}\OperatorTok{~}\NormalTok{dist)}
\end{Highlighting}
\end{Shaded}

\begin{figure}[H]

{\centering \includegraphics{paper_files/figure-latex/fig-3-1} 

}

\caption{Scatterplot of Speed and Distance}\label{fig:fig-3}
\end{figure}

\section{Good practices}\label{good-practices}

Every researcher has his own optimized setup. Currently I would
recommend the following:

\begin{itemize}
\tightlist
\item
  Keep all files of your project (that matter for producing the PDF) in
  one folder without subfolders. You can zip and directly upload that
  folder to the \href{https://dataverse.harvard.edu/}{Harvard
  dataverse}).
\item
  Make sure that filenames have a logic to them.

  \begin{itemize}
  \tightlist
  \item
    Main file with text/code: ``paper.rmd'', ``report.rmd''
  \item
    Data files: ``data\_xxxxxx.*"
  \item
    Image files: ``fig\_xxxxxx.*"
  \item
    Tables files: ``table\_xxxx.*"
  \item
    etc.
  \item
    Ideally, your filenames will correspond to the names in the paper.
    For instance, Figure 1 in the paper may have a corresponding file
    called \texttt{fig\_1\_xxxxx.pdf}.
  \end{itemize}
\item
  Use document the document outline in R studio (Ctrl + Shift + O) when
  you work with R Markdown.
\item
  Name rchunks according to what they do or produce:

  \begin{itemize}
  \tightlist
  \item
    ``\texttt{fig-...}'' for chunks producing figures
  \item
    ``\texttt{table-...}'' for chunks producing tables
  \item
    ``\texttt{model-...}'' for chunks producing model estimates
  \item
    ``\texttt{import-...}'' for chunks importing data
  \item
    ``\texttt{recoding-...}'' for chunks in which data is recoded
  \end{itemize}
\item
  Use ``really'' informative variable names:

  \begin{itemize}
  \tightlist
  \item
    Q: What do you think does the variable \emph{trstep} measure? It
    actually measures trust in the European parliament.

    \begin{itemize}
    \tightlist
    \item
      How could we call this variable instead? Yes,
      \texttt{trust.european.parliament} which is longer but will
      probably be understood by one of your descendants in 100 years.
    \end{itemize}
  \item
    If your setup is truly reproducible you will probably re-use the
    variable names that you generate as variable names in the tables you
    produce. Hence, there is an incentive to use good names.
  \end{itemize}
\item
  Use unique identifiers in the final document:

  \begin{itemize}
  \tightlist
  \item
    e.g., name the models you estimate ``M1'', ``M2'' etc.
  \item
    These unique names should also appear in the published paper.
  \item
    Think of someone who wants to produce Figure 1/Model 1 in your paper
    but doesn't find it in your code\ldots{}
  \end{itemize}
\end{itemize}

\section{Citation styles}\label{citation-styles}

If your study needs to follow a particular citation style, you can set
the corresponding style in the header of your \texttt{.rmd} document. To
do so you have to download the corresponding \texttt{.csl} file.

In the present document we use the style of the American Sociological
Association and set it in the preamble with
\texttt{csl:\ american-sociological-association.csl}. However, you also
need to download the respective \texttt{.csl} file from the following
github page: \url{https://github.com/citation-style-language/styles} and
copy it into your working directory for it to work.

The github directory contains a wide variety of citation style files
depending on what discipline your work in.

\section{Appendix}\label{appendix}

\subsection{All the code in the paper}\label{all-the-code-in-the-paper}

To simply attach all the code you used in the PDF file in the appendix
see the R chunk in the underlying \texttt{.rmd} file:

\begin{Shaded}
\begin{Highlighting}[]
\NormalTok{knitr}\OperatorTok{::}\NormalTok{opts_chunk}\OperatorTok{$}\KeywordTok{set}\NormalTok{(}\DataTypeTok{cache =} \OtherTok{FALSE}\NormalTok{)}
\CommentTok{# Use chache = TRUE if you want to speed up compilation}

\CommentTok{# A function to allow for showing some of the inline code}
\NormalTok{rinline <-}\StringTok{ }\ControlFlowTok{function}\NormalTok{(code)\{}
\NormalTok{  html <-}\StringTok{ '<code  class="r">``` `r CODE` ```</code>'}
  \KeywordTok{sub}\NormalTok{(}\StringTok{"CODE"}\NormalTok{, code, html)}
\NormalTok{\}}
\KeywordTok{install.packages}\NormalTok{(}\KeywordTok{c}\NormalTok{(}\StringTok{'tinytex'}\NormalTok{, }\StringTok{'rmarkdown'}\NormalTok{))}
\NormalTok{tinytex}\OperatorTok{::}\KeywordTok{install_tinytex}\NormalTok{()}
\KeywordTok{install.packages}\NormalTok{(}\KeywordTok{c}\NormalTok{(}\StringTok{"rmarkdown"}\NormalTok{, }\StringTok{"knitr"}\NormalTok{, }\StringTok{"kableExtra"}\NormalTok{,}
                   \StringTok{"stargazer"}\NormalTok{, }\StringTok{"plotly"}\NormalTok{, }\StringTok{"knitr"}\NormalTok{,}
                   \StringTok{"bookdown"}\NormalTok{))}
\KeywordTok{cat}\NormalTok{(}\KeywordTok{paste}\NormalTok{(}\StringTok{"#"}\NormalTok{, }\KeywordTok{capture.output}\NormalTok{(}\KeywordTok{sessionInfo}\NormalTok{()), }\StringTok{"}\CharTok{\textbackslash{}n}\StringTok{"}\NormalTok{, }\DataTypeTok{collapse =}\StringTok{""}\NormalTok{)) }
  \CommentTok{# or use message() instead of cat()}
\NormalTok{data <-}\StringTok{ }\KeywordTok{read.csv}\NormalTok{(}\StringTok{"data.csv"}\NormalTok{)}
\KeywordTok{head}\NormalTok{(data)}
\KeywordTok{dput}\NormalTok{(data)}
\NormalTok{data <-}\StringTok{ }\KeywordTok{structure}\NormalTok{(}\KeywordTok{list}\NormalTok{(}\DataTypeTok{X =} \DecValTok{1}\OperatorTok{:}\DecValTok{50}\NormalTok{, }\DataTypeTok{speed =} \KeywordTok{c}\NormalTok{(4L, 4L, 7L, 7L, 8L, 9L, 10L, }
\NormalTok{10L, 10L, 11L, 11L, 12L, 12L, 12L, 12L, 13L, 13L, 13L, 13L, 14L, }
\NormalTok{14L, 14L, 14L, 15L, 15L, 15L, 16L, 16L, 17L, 17L, 17L, 18L, 18L, }
\NormalTok{18L, 18L, 19L, 19L, 19L, 20L, 20L, 20L, 20L, 20L, 22L, 23L, 24L, }
\NormalTok{24L, 24L, 24L, 25L), }\DataTypeTok{dist =} \KeywordTok{c}\NormalTok{(2L, 10L, 4L, 22L, 16L, 10L, 18L, }
\NormalTok{26L, 34L, 17L, 28L, 14L, 20L, 24L, 28L, 26L, 34L, 34L, 46L, 26L, }
\NormalTok{36L, 60L, 80L, 20L, 26L, 54L, 32L, 40L, 32L, 40L, 50L, 42L, 56L, }
\NormalTok{76L, 84L, 36L, 46L, 68L, 32L, 48L, 52L, 56L, 64L, 66L, 54L, 70L, }
\NormalTok{92L, 93L, 120L, 85L)), }
\DataTypeTok{class =} \StringTok{"data.frame"}\NormalTok{, }\DataTypeTok{row.names =} \KeywordTok{c}\NormalTok{(}\OtherTok{NA}\NormalTok{, }
\OperatorTok{-}\NormalTok{50L))}
\KeywordTok{library}\NormalTok{(stargazer)}
\KeywordTok{stargazer}\NormalTok{(cars, }
          \DataTypeTok{title =} \StringTok{"Summary table with stargazer"}\NormalTok{,}
          \DataTypeTok{label=}\StringTok{"tab1"}\NormalTok{, }
          \DataTypeTok{table.placement =} \StringTok{"H"}\NormalTok{, }
          \DataTypeTok{header=}\OtherTok{FALSE}\NormalTok{)}
\KeywordTok{library}\NormalTok{(stargazer)}
\NormalTok{model1 <-}\StringTok{ }\KeywordTok{lm}\NormalTok{(speed }\OperatorTok{~}\StringTok{ }\NormalTok{dist, }\DataTypeTok{data =}\NormalTok{ cars)}
\NormalTok{model2 <-}\StringTok{ }\KeywordTok{lm}\NormalTok{(speed }\OperatorTok{~}\StringTok{ }\NormalTok{dist, }\DataTypeTok{data =}\NormalTok{ cars)}
\NormalTok{model3 <-}\StringTok{ }\KeywordTok{lm}\NormalTok{(dist }\OperatorTok{~}\StringTok{ }\NormalTok{speed, }\DataTypeTok{data =}\NormalTok{ cars)}
\KeywordTok{stargazer}\NormalTok{(model1, model2, model3,}
          \DataTypeTok{title =} \StringTok{"Regression table with stargazer"}\NormalTok{,}
          \DataTypeTok{label=}\StringTok{"tab2"}\NormalTok{, }
          \DataTypeTok{table.placement =} \StringTok{"H"}\NormalTok{, }
          \DataTypeTok{column.labels =} \KeywordTok{c}\NormalTok{(}\StringTok{"M1"}\NormalTok{, }\StringTok{"M2"}\NormalTok{, }\StringTok{"M3"}\NormalTok{),}
          \DataTypeTok{model.numbers =} \OtherTok{FALSE}\NormalTok{,}
          \DataTypeTok{header=}\OtherTok{FALSE}\NormalTok{)}
\KeywordTok{library}\NormalTok{(knitr)}
\KeywordTok{library}\NormalTok{(kableExtra)}
\KeywordTok{kable}\NormalTok{(cars[}\DecValTok{1}\OperatorTok{:}\DecValTok{10}\NormalTok{,], }\DataTypeTok{row.names =} \OtherTok{TRUE}\NormalTok{, }
      \DataTypeTok{caption =} \StringTok{'Table with kable() and kablestyling()'}\NormalTok{, }
      \DataTypeTok{format =} \StringTok{"latex"}\NormalTok{, }\DataTypeTok{booktabs =}\NormalTok{ T) }\OperatorTok
\StringTok{        }\KeywordTok{kable_styling}\NormalTok{(}\DataTypeTok{full_width =}\NormalTok{ T, }
                      \DataTypeTok{latex_options =} \KeywordTok{c}\NormalTok{(}\StringTok{"striped"}\NormalTok{, }
                                        \StringTok{"scale_down"}\NormalTok{,}
                                        \StringTok{"HOLD_position"}\NormalTok{),}
                      \DataTypeTok{font_size =} \DecValTok{10}\NormalTok{)}
\KeywordTok{plot}\NormalTok{(cars}\OperatorTok{$}\NormalTok{speed, cars}\OperatorTok{$}\NormalTok{dist)}
\NormalTok{mtcars}\OperatorTok{$}\NormalTok{cyl <-}\StringTok{ }\KeywordTok{as.factor}\NormalTok{(mtcars}\OperatorTok{$}\NormalTok{cyl) }\CommentTok{# Convert cyl to factor}
\KeywordTok{library}\NormalTok{(ggplot2)}
\KeywordTok{ggplot}\NormalTok{(mtcars, }\KeywordTok{aes}\NormalTok{(}\DataTypeTok{x=}\NormalTok{wt, }\DataTypeTok{y=}\NormalTok{mpg, }\DataTypeTok{shape=}\NormalTok{cyl)) }\OperatorTok{+}\StringTok{ }\KeywordTok{geom_point}\NormalTok{() }\OperatorTok{+}
\StringTok{  }\KeywordTok{labs}\NormalTok{(}\DataTypeTok{x=}\StringTok{"Weight (lb/1000)"}\NormalTok{, }\DataTypeTok{y =} \StringTok{"Miles/(US) gallon"}\NormalTok{, }
       \DataTypeTok{shape=}\StringTok{"Number of }\CharTok{\textbackslash{}n}\StringTok{ Cylinders"}\NormalTok{) }\OperatorTok{+}\StringTok{ }\KeywordTok{theme_classic}\NormalTok{()}
\KeywordTok{library}\NormalTok{(plotly)}
\KeywordTok{plot_ly}\NormalTok{(cars, }\DataTypeTok{type =} \StringTok{"scatter"}\NormalTok{, }\DataTypeTok{mode=}\StringTok{"markers"}\NormalTok{,}
        \DataTypeTok{x=}\OperatorTok{~}\NormalTok{speed, }
        \DataTypeTok{y=}\OperatorTok{~}\NormalTok{dist)}
\end{Highlighting}
\end{Shaded}

\section*{References}\label{references}
\addcontentsline{toc}{section}{References}

\hypertarget{refs}{}
\hypertarget{ref-markdown2017}{}
Allaire, JJ, Jeffrey Horner, Vicent Marti, and Natacha Porte. 2017.
\emph{Markdown: 'Markdown' Rendering for R}.

\hypertarget{ref-Bauer2018-hd}{}
Bauer, Paul C. 2018. ``Unemployment, Trust in Government, and
Satisfaction with Democracy: An Empirical Investigation.'' \emph{Socius}
4:1--14.

\hypertarget{ref-hlavac2013stargazer}{}
Hlavac, Marek. 2013. ``Stargazer: LaTeX Code and Ascii Text for
Well-Formatted Regression and Summary Statistics Tables.'' \emph{URL:
Http://CRAN. R-Project. Org/Package= Stargazer}.

\hypertarget{ref-Kirsop2005-ro}{}
Kirsop, Barbara and Leslie Chan. 2005. ``Transforming Access to Research
Literature for Developing Countries.'' \emph{Serials Review}
31(4):246--55.

\hypertarget{ref-R2017}{}
R Core Team. 2017. \emph{R: A Language and Environment for Statistical
Computing}. Vienna, Austria: R Foundation for Statistical Computing.

\hypertarget{ref-Rstudio2015}{}
RStudio Team. 2015. \emph{RStudio: Integrated Development Environment
for R}. Boston, MA: RStudio, Inc.

\hypertarget{ref-plotly}{}
Sievert, Carson, Chris Parmer, Toby Hocking, Scott Chamberlain, Karthik
Ram, Marianne Corvellec, and Pedro Despouy. 2017. \emph{Plotly: Create
Interactive Web Graphics via 'Plotly.js'}.

\hypertarget{ref-knitr3}{}
Xie, Yihui. 2014. ``Knitr: A Comprehensive Tool for Reproducible
Research in R.'' in \emph{Implementing reproducible computational
research}, edited by V. Stodden, F. Leisch, and R. D. Peng. Chapman;
Hall/CRC.

\hypertarget{ref-knitr2}{}
Xie, Yihui. 2015. \emph{Dynamic Documents with R and Knitr}. 2nd ed.
Boca Raton, Florida: Chapman; Hall/CRC.

\hypertarget{ref-bookdown2}{}
Xie, Yihui. 2016. \emph{Bookdown: Authoring Books and Technical
Documents with R Markdown}. Boca Raton, Florida: Chapman; Hall/CRC.

\hypertarget{ref-bookdown1}{}
Xie, Yihui. 2017. \emph{Bookdown: Authoring Books and Technical
Documents with R Markdown}.

\hypertarget{ref-knitr1}{}
Xie, Yihui. 2018a. \emph{Knitr: A General-Purpose Package for Dynamic
Report Generation in R}.

\hypertarget{ref-tinytex}{}
Xie, Yihui. 2018b. \emph{Tinytex: Helper Functions to Install and
Maintain 'Tex Live', and Compile 'Latex' Documents}.

\hypertarget{ref-kableextra}{}
Zhu, Hao. 2017. \emph{KableExtra: Construct Complex Table with 'Kable'
and Pipe Syntax}.


\end{document}
